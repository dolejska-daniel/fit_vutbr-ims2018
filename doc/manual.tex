\documentclass[11pt]{article}
\usepackage[utf8]{inputenc}
\usepackage[T1]{fontenc}
\usepackage[main=czech]{babel}
\usepackage[nottoc]{tocbibind}
\usepackage{url}
\usepackage{xcolor}
\usepackage{minted}
\usepackage{array}

\usepackage{parskip}
\setlength{\parindent}{20pt}

\title{\textbf{FIT VUT Brno - IMS}\\
	2. Doprava zboží nebo osob\\
	Správa vícepatrového skladu}
\author{Daniel Dolejška, \texttt{xdolej08@stud.fit.vutbr.cz}}
\date{\today}

\begin{document}
	
	\maketitle
	\tableofcontents
	
	
	\newpage
	
	\section{Úvod}
	% Úvod musí vysvětlit, proč se celá práce dělá a
	%proč má uživatel výsledků váš dokument číst.
	%  Motivace a širší souvislosti. Kontext problému.
	%  V této práci je řešena implementace ..., která bude použita pro
	%sestavení modelu ...
	%  Na základě modelu a simulačních experimentů bude ukázáno
	%chování systému ... v podmínkách ...
	%  Smyslem experimentů je demonstrovat, že pokud by ..., pak by
	%...
	%  Správnost zvolené koncepce byla ověřena...
	% Psaní úvodů je náročná práce. Úvody se čtou!!!
	Tato práce se zaobírá modelováním vícepatrového skladu.
	Klíčovým prvkem tohoto systému je pohyb palet se zbožím mezi jednotlivými patry.
	
	Přesun palet mezi patry je možný pouze za pomoci několika vysokozdvižných vozíků, které se ve skladě v průběhu dne pohybují.
	Pro účely přesunu palet mezi patry je vyhrazen jeden vysokozdvižný vozík.
	
	S vozíky má z bezpečnostních důvodů oprávnění pracovat pouze určité procento zaměstnanců.
	Cílem této práce je zjistit následující:
	\begin{itemize}
		\item kolik času je ztraceno v průběhu přesunu palety mezi patry
		\item průměrné vytížení vozíků přesunem palet mezi patry
		\item zvýšení efektivity při nahrazení vozíků výtahem
	\end{itemize}

	\subsection{Zdroje faktů}
	%1.1 Zdroje faktů
	% Kdo se na práci podílel jako autor, odborný
	%konzultant, dodavatel odborných faktů,
	%  význačné zdroje literatury/fakt, ...
	%  je ideální, pokud jste vaši koncepci konzultovali s nějakou
	%autoritou v oboru (v IMS projektu to je hodnoceno, ovšem
	%není vyžadováno)
	%  pokud nebudete mít odborného konzultanta, nevadí. Nelze
	%ovšem tvrdit, že jste celé dílo vymysleli s nulovou interakcí s
	%okolím a literaturou.
	% Zdroj údajů
	Fakta o tomto systému jsou původem z několika zdrojů - prvním je osobní zkušenost a pozorování v době, kdy jsem ve skladě pracoval.
	Mezi další zdroje informací a faktů pak patří zaměstnanci firmy, pracující na skladě.


	\subsection{Ověření validity}
	%1.2 Ověření validity/funkčnosti
	% V jakém prostředí a za jakých podmínek
	%probíhalo experimentální ověřování validity
	%modelu.
	% Pokud čtenář/zadavatel vaší zprávy neuvěří ve
	%validitu vašeho modelu, obvykle vaši práci
	%odmítne už v tomto okamžiku.
	% Alternativně lze zmínit v závěru experimentů.

	\newpage
	\section{Fakta}
	% Podstatná fakta o systému musí být zdůvodněna a
	%podepřena důvěryhodným zdrojem (vědecký článek,
	%kniha, osobní měření a zjišťování). Alespoň jeden (lépe 2)
	%zdroj.
	% Fakta:
	%  Kterékoliv číslo, fakt, stav, vztah.
	%  Za každým takovým údajem musí následovat odkaz na zdroj (1
	%důvěryhodný nebo několik jiných).
	%  Hypotézy/předpoklady (podklady).
	% SHO: proces příchodů požadavků/doby obsluhy,
	%struktura systému, ...

	%Věrohodnost faktů
	% Zdroj literatury — články, normy, tech.
	%dokumenty.
	% Osobní/zprostředkované pozorování v terénu
	%— má však svoje náležitosti (např. měření
	%stochastického jevu).
	% Hypotézy — založené na podobném systému.
	\footnotetext[1]{Zdroj: Osobní měření}
	\footnotetext[2]{Zdroj: Zaměstnanec firmy}
	\begin{enumerate}
		\item Sklad se nachází na třech podlažích \footnotemark[1]
		\begin{itemize}
			\item \textsf{Přízemí} - práce s vysokozdvižnými vozíky není nutná
			\item \textsf{1. a 2. podlaží} - paletu je nutné vždy dostat o patro níže
		\end{itemize}
		\item Na každém podlaží se nachází omezené množství prázdných palet \footnotemark[1]
		\item Ve skladě se aktivně pohybují průměrně 4 vysokozdvižné vozíky \footnotemark[2]
		\item Ve skladě běžně pracuje průměrně 15 lidí \footnotemark[2]
		% 6-9hod – 8,
		% 9-15hod – 20,
		% 15-18hod – 8
		\begin{itemize}
			\item \textsf{75\%} může operovat vysokozdvižný vozík
			\item \textsf{25\%} musí požádat jiného zaměstnance o přesun palety
		\end{itemize}
		\item Za den je průměrně zpracováno (vyskladněno) 410 objednávek \footnotemark[2]
		\begin{itemize}
			\item \textsf{75\%} začíná ve 2. patře, následně pokračuje do prvního patra
			\item \textsf{20\%} začíná v 1. patře, následně pokračuje do přízemí
			\item \textsf{5\%} pouze z přízemí
		\end{itemize}
	\end{enumerate}
	

	\section{Koncepce}
	%3. Koncepce modelu/simulátoru
	%  Konceptuální model je abstrakce reality a redukce reality
	%na soubor relevantních faktů pro sestavení simulačního
	%modelu.
	% Pokud některé partie reality zanedbáváte nebo
	%zjednodušujete, musí to být zdůvodněno a v
	%ideálním případě musí být prokázáno, že to
	%neovlivní validitu modelu.
	% Výsledek kapitoly: konceptuální (abstraktní)
	%model s vyznačením relevantních faktů.
	% Základní koncept modelu.

	%Fakta versus Koncepce
	% Fakta: soupis znalostí o daném problému.
	% Koncepce:
	%  převzetí faktů do modelu,
	%  zdůvodněné provedené zjednodušení faktů,
	%  abstraktní popis modelu/programu.
	% Těžiště modelářské práce. Vytváříme abstraktní
	%model.
	% Návod: koncepci vaší práce MUSÍ pochopit
	%libovolný technik (a často i manažer...).

	%Koncepce SHO: Petriho síť
	% Může být dobré ukázat vztahy mezi procesy a
	%zdroji v systému.
	%  Neberme AM v Petriho síti jako detailní program.
	%AM má ukázat zásadní fakta o systému. Části, které v PS
	%nelze vyjádřit, vyjádřete slovním popisem.
	%  Konkrétní fronta se volí podle.... Vygeneruje se normal(X,Y)
	%značek...
	% AM musí být stručný, přehledný a srozumitelný
	%Může být čitelně psán rukou + scan.

	%Zjišťování faktů
	% Náročná práce, mnohdy téměř partyzánská.
	% Literatura.
	% Osobní (nedestruktivní) zjišťování v terénu.
	% Je to součást modelářské práce.

	\section{Způsob řešení}

	\section{Experimenty}
	%Simulační studie začíná formulováním problému:
	%  co chci zjistit,
	%  (proč je k tomu potřeba simulační model.)
	% Studie končí výslovením závěru:
	%  co jsem tedy zjistil,
	%  co bych ještě mohl zjistit,
	%  (proč by to nešlo bez modelu.)
	%  Bez experimentů práce nedává smysl!

	%Experimenty: úvod
	% Experimentování musí mít předem zvolený a zdůvodněný
	%řád, či postup.
	% Okolnosti experimentování:
	%  datová sada, konfigurace měřící aparatury, …
	%  závislost Y na X (graf).
	% Test versus Experiment.
	%  “měření” != experiment !!!
	% Experimenty se i ladí model - kalibrační experimenty.
	%  ... na základě tohoto experimentu jsme korigovali parametr x..
	%  u praktických simulačních studií se nepublikuje.
	%Struktura kapitoly Experimenty
	% 5.1 Postup experimentování a okolnosti studie
	% 5.2 Dokumentace jednotlivých experimentů
	% 5.3 Závěr experimentů
	% Poznámka: experimentování je činnost
	%vyžadující preciznost.
	%  modelování a SIMULACE
	%Dokumentace experimentu
	% Protokolární forma:
	%  vstupy a okolnosti,
	%  výstupy a pozorování,
	%  interpretace výsledků.
	% Interpretace výsledků:
	%  Rozbor výsledků: co v nich má čtenář vidět.
	%  Grafy mají pojmenované a kalibrované osy.
	% Návrh dalšího experimentu.
	%Závěry experimentů
	% Co bylo experimentováním zjištěno.
	% Jaké chyby v modelu byly odstraněny (oproti
	%původním předpokladům ... došlo ke změně
	%koncepce ... protože ...).
	% Co lze zjistit dalšími experimenty.

	\section{Závěr}
	%Jednoznačná odpověď na prvotní otázku studie.
	%  Studií provedenou na našem modelu bylo jednoznačně
	%prokázáno/vyvráceno, že ...
	%  V rámci experimentů bylo zjištěno, že průměrné zatížení ...
	%je ...
	%  Z experimentů vyplývá jednoznačné doporučení, aby
	%provozovatel ... rozšířil výrobu o ...
	%  Ze statisticky zpracovaného měření v terénu plyne, že proces
	%příchodů ... se řídí normálním rozložením se středem a ....
	%  Na přiložených demo-příkladech jsme ověřili funkčnost ...

	
	\newpage
	\bibliographystyle{plain}
	\bibliography{zdroje}
	
\end{document}
